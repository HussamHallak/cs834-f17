\section*{Question 5:}
Exercise 8.9:
For one query in the CACM collection, generate a ranking and calculate BPREF. Show that the two formulations of BPREF give the same value.
\subsection*{Answer:}

I used query 13 that I used for question 2. The steps to find the first 10 results for the query and the relevant/non-relevant documents are identical to the steps in question 2.

\textbf{Results for query \#13}:

\begin{longtable}{ |p{1cm}|p{4cm}|p{3cm}| } 
\hline
Result & File Name & Hit/Miss \\
\hline
1 & CACM-2897.html & Hit \\
\hline
2 & CACM-2033.html & Miss \\
\hline
3 & CACM-1947.html & Hit \\
\hline
4 & CACM-2559.html & Miss \\
\hline
5 & CACM-2491.html & Miss \\
\hline
6 & CACM-2680.html & Miss \\
\hline
7 & CACM-2701.html & Miss \\
\hline
8 & CACM-1886.html & Miss \\
\hline
9 & CACM-2944.html & Miss  \\
\hline
10 & CACM-2537.html & Miss \\
\hline
\end{longtable}

It's time to calculate BPREF by hand using the two definitions of BPREF:

$$ BPREF = \frac{1}{R} \sum_{d_r} (1-\frac{{N_d}_r}{R}) $$

And

$$ BPREF = \frac{P}{P+Q} $$

Where R is the number of relevant documents, $d_r$ is a relevant document, and ${N_d}_r$ gives the number of non-relevant documents (from the set of R non-relevant documents that are considered) that are ranked higher than $d_r$ 

R = 2, therefore the number of non-relevant documents that are considered is 2. 

$$ BPREF = \frac{1}{2} [(1-0/2) + (1-1/2)] = \frac{3}{4} = 0.75 $$

Using the alternative definition:

$$ BPREF = \frac{2}{2+1} = \frac{2}{3} = 0.67 $$

I have different values at the end. I tried different definitions that I found online, but came up with results that are much different than this.